%%%%%%%%%%%%%%%%%%%%%%%%%%%%%%%%%%%%%%%%%%%%%%%%%%%%%%%%%%%%%%%%%%%%%%%
% This document is based on the template: Large Colored Title Article %
%                                         Version 1.1 (25/11/12)      %
%                                                                     %
% The template was downloaded from: http://www.LaTeXTemplates.com     %
%                                                                     %
% Original author:                                                    %
% Frits Wenneker (http://www.howtotex.com)                            %
%                                                                     %
% License:                                                            %
% CC BY-NC-SA 3.0 (http://creativecommons.org/licenses/by-nc-sa/3.0/) %
%                                                                     %
% Authors of this version:                                             %
% Adrian Quintas Garcia              %
% Yuriy Mischenko              %
%                                                                     %
% Original licensing terms are maintained                             %
%%%%%%%%%%%%%%%%%%%%%%%%%%%%%%%%%%%%%%%%%%%%%%%%%%%%%%%%%%%%%%%%%%%%%%%

%----------------------------------------------------------------------------------------
%	PACKAGES AND OTHER DOCUMENT CONFIGURATIONS
%----------------------------------------------------------------------------------------

\documentclass[DIV=calc,paper=a4,fontsize=11pt,onecolumn]{scrartcl}	 % A4 paper and 11pt font size

\usepackage[galician]{babel} % Galician language/hyphenation
\usepackage[utf8]{inputenc}
\usepackage[protrusion=true,expansion=true]{microtype} % Better typography
\usepackage{amsmath,amsfonts,amsthm} % Math packages
\usepackage[svgnames]{xcolor} % Enabling colors by their 'svgnames'
\usepackage[hang,small,labelfont=bf,up,textfont=it,up]{caption} % Custom captions under/above floats in tables or figures
\usepackage{booktabs} % Horizontal rules in tables
\usepackage{fix-cm}	 % Custom font sizes - used for the initial letter in the document

\usepackage{sectsty} % Enables custom section titles
\allsectionsfont{\usefont{OT1}{phv}{b}{n}} % Change the font of all section commands

\usepackage{fancyhdr} % Needed to define custom headers/footers
\pagestyle{fancy} % Enables the custom headers/footers
\usepackage{lastpage} % Used to determine the number of pages in the document (for "Page X of Total")

% Headers - all currently empty
\lhead{}
\chead{}
\rhead{}

% Footers
\lfoot{\textsc{vvs-estratexia-probas}}
\cfoot{}
\rfoot{\footnotesize Páxina \thepage\ de \pageref{LastPage}} % "Page 1 of 2"

\renewcommand{\headrulewidth}{0.0pt} % No header rule
\renewcommand{\footrulewidth}{0.4pt} % Thin footer rule

\definecolor{UDC}{RGB}{206,0,124}
\definecolor{DarkUDC}{rgb}{0.75,0.75,0.75}
\definecolor{LightUDC}{RGB}{128,128,128}

\usepackage{lettrine} % Package to accentuate the first letter of the text
\newcommand{\initial}[1]{ % Defines the command and style for the first letter
\lettrine[lines=3,lhang=0.3,nindent=0em]{
\color{UDC}
{\textsf{#1}}}{}}

%----------------------------------------------------------------------------------------
%	TITLE SECTION
%----------------------------------------------------------------------------------------

\usepackage{titling} % Allows custom title configuration

\newcommand{\HorRule}{\color{UDC} \rule{\linewidth}{1pt}} % Defines the pink horizontal rule around the title

\pretitle{\vspace{-30pt} \begin{flushleft} \HorRule \fontsize{20}{20} \usefont{OT1}{phv}{b}{n} \color{DarkUDC} \selectfont} % Horizontal rule before the title

\title{ESTRATEXIA DE PROBAS} % Your article title

\posttitle{\par\end{flushleft}\vskip 0.5em} % Whitespace under the title

\preauthor{\begin{flushleft}\large \lineskip 0.5em \usefont{OT1}{phv}{b}{sl} \color{DarkUDC}} % Author font configuration

\author{Título proxecto: Práctica \\
        Ref. proxecto: Verificación y validación de software}

\postauthor{\footnotesize \usefont{OT1}{phv}{m}{sl} \color{Black} % Configuration for the institution name
\par\end{flushleft}\HorRule} % Horizontal rule after the title

\date{\sffamily Validación e Verificación de Software} % Add a date here if you would like one to appear underneath the title block

%----------------------------------------------------------------------------------------

\usepackage{graphicx}
\usepackage{hyperref}
\hypersetup{colorlinks=true,
            allcolors=UDC}

\usepackage{array}
\usepackage{colortbl}

%----------------------------------------------------------------------------------------

\newcommand{\hint}[1]{\begin{quote}\itshape #1 \end{quote}}

%----------------------------------------------------------------------------------------

\begin{document}

\maketitle % Print the title
\thispagestyle{fancy} % Enabling the custom headers/footers for the first page 

%----------------------------------------------------------------------------------------
%	ABSTRACT
%----------------------------------------------------------------------------------------

\vspace*{1cm}

\begin{center}
\small \sffamily
\begin{tabular}{c|c|c}
Data de aprobación & Control de versións & Observacións \\ \hline
& & \\ \hline
& & \\ \hline
& & \\
\end{tabular}
\end{center}

\clearpage

%----------------------------------------------------------------------------------------
%	ARTICLE CONTENTS
%----------------------------------------------------------------------------------------

\section{Introdución}

\subsection{Propósito do documento}

\hint{El proyecto se comporta correctamente a un conjunto de test que evalúan funcionalidades de la lógica de la práctica. Durante el desarrollo no se han encontrado problemas, siendo todos los tests satisfactorios. Sin embargo, ante un posible error se creará un issue en el repositorio, que será asignado a alguno de los miembros del equipo, para el desarrollo de un fix que solucione cualquier bug encontrado.}

\subsection{Alcance e difusión}

\hint{El documento de pruebas estará disponible en el repositorio del proyecto, y estará accesible para cualquiera que quiera ejecutar el plan de pruebas.}

\subsection{Contexto}

\hint{Este documento explicará la estrategia de plan de pruebas para el proyecto.}

\section{Situación actual}

\hint{El proyecto corre con uno o varios servidores, que tienen pre-configurado un puerto en el cual estarán escuchando peticiones.
}

\section{Análise de problemas}

\hint{Listado de posibles problemas.}

\subsection{Problema$_1$}

\hint{Es posible que los puertos de los servidores estén ocupados por otro proceso, por eso antes de ejecutar las pruebas es necesario asegurarse que los puertos configurados dentro de la aplicación estén disponibles. Es posible que ante una terminación inesperada del proceso, éste quede ocupando el puerto con el cual arrancó en un principio.}

\subsubsection{Solución$_{11}$}

\hint{Si un proceso está ocupando el puerto, éste puede terminarse para liberarlo.}

\subsubsection{Solución$_{12}$}

\hint{Si el puerto estuviese ocupado y no pudiese ser liberado, se debería cambiar la configuración de la aplicación para que los servidores se ejecutasen en otros puertos diferentes.}

\section{Criterios de diagnóstico solución-problema}

\hint{Puede considerar como problema solventado si el proceso arranca correctamente, y los servidores inician la escucha en el puerto configurado.}

\section{Actuacións recomendadas}

\hint{La primera solución para el problema es la recomendada.}

\section{Política de revisión}

\hint{Ante cualquier cambio en la arquitectura, cambio de lógica de de funcionalidades, o adicción de una nueva implementación, este documento debe ser revisado para mantener actualizada la estrategia de pruebas.}

\section{Outros aspectos de interese}

\hint{No procede.}

\end{document}
