%%%%%%%%%%%%%%%%%%%%%%%%%%%%%%%%%%%%%%%%%%%%%%%%%%%%%%%%%%%%%%%%%%%%%%%
% This document is based on the template: Large Colored Title Article %
%                                         Version 1.1 (25/11/12)      %
%                                                                     %
% The template was downloaded from: http://www.LaTeXTemplates.com     %
%                                                                     %
% Original author:                                                    %
% Frits Wenneker (http://www.howtotex.com)                             %
%                                                                     %
% License:                                                            %
% CC BY-NC-SA 3.0 (http://creativecommons.org/licenses/by-nc-sa/3.0/) %
%                                                                     %
% Authors of this version:                                             %
% Adrian Quintas Garcia              %
% Yuriy Mischenko              %
%                                                                     %
% Original licensing terms are maintained                             %
%%%%%%%%%%%%%%%%%%%%%%%%%%%%%%%%%%%%%%%%%%%%%%%%%%%%%%%%%%%%%%%%%%%%%%%

%----------------------------------------------------------------------------------------
%	PACKAGES AND OTHER DOCUMENT CONFIGURATIONS
%----------------------------------------------------------------------------------------

\documentclass[DIV=calc,paper=a4,fontsize=11pt,onecolumn]{scrartcl}	 % A4 paper and 11pt font size

\usepackage[galician]{babel} % Galician language/hyphenation
\usepackage[utf8]{inputenc}
\usepackage[protrusion=true,expansion=true]{microtype} % Better typography
\usepackage{amsmath,amsfonts,amsthm} % Math packages
\usepackage[svgnames]{xcolor} % Enabling colors by their 'svgnames'
\usepackage[hang,small,labelfont=bf,up,textfont=it,up]{caption} % Custom captions under/above floats in tables or figures
\usepackage{booktabs} % Horizontal rules in tables
\usepackage{fix-cm}	 % Custom font sizes - used for the initial letter in the document

\usepackage{sectsty} % Enables custom section titles
\allsectionsfont{\usefont{OT1}{phv}{b}{n}} % Change the font of all section commands

\usepackage{fancyhdr} % Needed to define custom headers/footers
\pagestyle{fancy} % Enables the custom headers/footers
\usepackage{lastpage} % Used to determine the number of pages in the document (for "Page X of Total")

% Headers - all currently empty
\lhead{}
\chead{}
\rhead{}

% Footers
\lfoot{\textsc{vvs-plan-probas}}
\cfoot{}
\rfoot{\footnotesize Páxina \thepage\ de \pageref{LastPage}} % "Page 1 of 2"

\renewcommand{\headrulewidth}{0.0pt} % No header rule
\renewcommand{\footrulewidth}{0.4pt} % Thin footer rule

\definecolor{UDC}{RGB}{206,0,124}
\definecolor{DarkUDC}{rgb}{0.75,0.75,0.75}
\definecolor{LightUDC}{RGB}{128,128,128}

\usepackage{lettrine} % Package to accentuate the first letter of the text
\newcommand{\initial}[1]{ % Defines the command and style for the first letter
\lettrine[lines=3,lhang=0.3,nindent=0em]{
\color{UDC}
{\textsf{#1}}}{}}

%----------------------------------------------------------------------------------------
%	TITLE SECTION
%----------------------------------------------------------------------------------------

\usepackage{titling} % Allows custom title configuration

\newcommand{\HorRule}{\color{UDC} \rule{\linewidth}{1pt}} % Defines the pink horizontal rule around the title

\pretitle{\vspace{-30pt} \begin{flushleft} \HorRule \fontsize{20}{20} \usefont{OT1}{phv}{b}{n} \color{DarkUDC} \selectfont} % Horizontal rule before the title

\title{PLAN DE PROBAS} % Your article title

\posttitle{\par\end{flushleft}\vskip 0.5em} % Whitespace under the title

\preauthor{\begin{flushleft}\large \lineskip 0.5em \usefont{OT1}{phv}{b}{sl} \color{DarkUDC}} % Author font configuration

\author{Título proxecto: Práctica \\
	Ref. proxecto: Verificación y validación de software}

\postauthor{\footnotesize \usefont{OT1}{phv}{m}{sl} \color{Black} % Configuration for the institution name
\par\end{flushleft}\HorRule} % Horizontal rule after the title

\date{\sffamily Validación e Verificación de Software} % Add a date here if you would like one to appear underneath the title block

%----------------------------------------------------------------------------------------

\usepackage{graphicx}
\usepackage{hyperref}
\hypersetup{colorlinks=true,
            allcolors=UDC}

\usepackage{array}
\usepackage{colortbl}

%----------------------------------------------------------------------------------------

\newcommand{\hint}[1]{\begin{quote}\itshape #1 \end{quote}}

%----------------------------------------------------------------------------------------

\begin{document}

\maketitle % Print the title
\thispagestyle{fancy} % Enabling the custom headers/footers for the first page 

%----------------------------------------------------------------------------------------
%	ABSTRACT
%----------------------------------------------------------------------------------------

\vspace*{1cm}

\begin{center}
\small \sffamily
\begin{tabular}{c|c|c}
Data de aprobación & Control de versións & Observacións \\ \hline
& & \\ \hline
& & \\ \hline
& & \\
\end{tabular}
\end{center}

\clearpage

%----------------------------------------------------------------------------------------
%	ARTICLE CONTENTS
%----------------------------------------------------------------------------------------

\section{Introdución}

\subsection{Propósito}

\hint{El plan de pruebas se aplicará a la segunda y última iteración de este proyecto.}

\section{Compoñentes avaliados}

\hint{Se someterán a pruebas tanto el servidor como contenido.}

\section{Funcionalidades}

\hint{
		\begin{enumerate}
			\item Alta de servidor
			\item Baja de servidor
			\item Buscar contenido en el servidor
			\item Agregar contenido en el servidor			
			\item Eliminar contenido en el servidor			
			\item Obtener lista de reproducción de contenido
			\item Buscar contenido dentro de emisoras
			\item Eliminar contenido dentro de emisoras
		\end{enumerate}
	}

\section{Interfaces}

\hint{El sistema no realizará ninguna interacción con otros componentes o sistemas.}

\section{Parámetros de calidade}

\hint{
	\begin{enumerate}
		\item Seguridad (apartado 4.2.1 Línea base)
		\item Fiabilidad (apartado 4.2.2 Línea base)
		\item Rendimiento (apartado 4.2.3 Línea base)
	\end{enumerate}
	}

\section{Especificación de probas}

\subsection{Servidor}

\hint{
		\begin{enumerate}
			\item Alta de servidor
			\subitem Petición de alta en el servidor
			\subitem Entrada: Petición GET
			\subitem Salida: Respuesta HTTP 200 OK
			\subitem Èxito: Recepción de salida esperada y creación del token en el servidor
			
			\item Baja de servidor
			\subitem Petición de baja en el servidor
			\subitem Entrada: Petición POST incluyendo el token en el cuerpo
			\subitem Salida: Respuesta HTTP 200 OK
			\subitem Èxito: Recepción de salida esperada y el token eliminado del servidor
			
			\item Peticiòn desconocida
			\subitem Petición a un path desconocido en el servidor
			\subitem Entrada: Cualquier path inexistente en el servidor
			\subitem Salida: Respuesta HTTP 404 NOT FOUND
			\subitem Èxito: Recepción de salida esperada
		
			\item Baja de servidor sin token
			\subitem Petición de baja al servidor sin un token proporcionado
			\subitem Entrada: Petición POST sin el token en el cuerpo
			\subitem Salida: Respuesta HTTP 400 BAD REQUEST
			\subitem Èxito: Recepción de salida esperada
			
			\item Agregar contenido
			\subitem Añadir un contenido al servidor
			\subitem Entrada: Petición POST con el token y contenido añadido en el cuerpo
			\subitem Salida: Respuesta HTTP 200 OK
			\subitem Èxito: Recepción de salida esperada y contenido añadido al servidor
			
			\item Agregar contenido sin token
			\subitem Añadir un contenido al servidor
			\subitem Entrada: Petición POST sin token y con el contenido añadido en el cuerpo
			\subitem Salida: Respuesta HTTP 403 FORBIDDEN
			\subitem Èxito: Recepción de salida esperada y sin cambios en el servidor
			
			\item Buscar contenido sin obtener publicidad
			\subitem Buscar contenido con un token válido que no obtenga publicidad
			\subitem Entrada: Petición POST con el token y el criterio de búsqueda en el cuerpo
			\subitem Salida: Respuesta HTTP 200 OK
			\subitem Èxito: Recepción de salida esperada y el resultado de la búsqueda, sin publicidad añadida
			
			\item Buscar contenido con publicidad
			\subitem Buscar contenido sin token, obtenido los resultados con publicidad
			\subitem Entrada: Petición POST con el criterio de búsqueda en el cuerpo
			\subitem Salida: Respuesta HTTP 200 OK
			\subitem Èxito: Recepción de salida esperada y el resultado de la búsqueda, con publicidad añadida
			
			\item Eliminar contenido
			\subitem Eliminar contenido enviando una petición al servidor con un token válido
			\subitem Entrada: Petición POST con el token y el contenido a eliminar
			\subitem Salida: Respuesta HTTP 200 OK
			\subitem Èxito: Recepción de salida esperada, eliminándose el contenido de la petición de los datos del servidor
			
			\item Buscar contenido con token caducado
			\subitem Realizar una petición de contenido con un token caducado
			\subitem Entrada: Petición POST con el token y el contenido a eliminar
			\subitem Salida: Respuesta HTTP 200 OK
			\subitem Èxito: Recepción de salida esperada, con el contenido incluyendo publicidad
		\end{enumerate}
	}
	
	\subsection{Contenido}
	
	\hint{
		\begin{enumerate}
			\item Obtener lista de reproducción de anuncio
			\subitem Petición lista de reproducción a un contenido de tipo anuncio
			\subitem Entrada: Petición de lista de reproducción a un contenido
			\subitem Salida: Lista de reproducción incluyendo el propio anuncio
			\subitem Èxito: Recepción de salida esperada
			
			\item Obtener lista de reproducción de una canción
			\subitem Petición lista de reproducción a un contenido de tipo canción
			\subitem Entrada: Petición de lista de reproducción a un contenido
			\subitem Salida: Lista de reproducción incluyendo a la propia canción
			\subitem Èxito: Recepción de salida esperada			
			
			\item Obtener lista de reproducción de una emisora
			\subitem Petición lista de reproducción a un contenido de tipo emisora
			\subitem Entrada: Petición de lista de reproducción a una emisora
			\subitem Salida: Lista de reproducción del contenido de la emisora
			\subitem Èxito: Recepción de salida esperada		
			
			\item Buscar contenido en un contenido tipo anuncio
			\subitem Búsqueda de contenido en un contenido de tipo anuncio
			\subitem Entrada: Petición del contenido a buscar
			\subitem Salida: Lista de contenidos incluyendo solamente el propio anuncio
			\subitem Èxito: Recepción de salida esperada							
			
			\item Buscar contenido en un contenido tipo canción
			\subitem Búsqueda de contenido en un contenido de tipo canción
			\subitem Entrada: Petición del contenido a buscar
			\subitem Salida: Lista de contenidos incluyendo solamente la propia canción
			\subitem Èxito: Recepción de salida esperada			
			
			\item Buscar contenido en un contenido tipo emisora
			\subitem Búsqueda de contenido en un contenido de tipo emisora
			\subitem Entrada: Petición del contenido a buscar
			\subitem Salida: Lista de contenidos de la emisora
			\subitem Èxito: Recepción de salida esperada				
			
			\item Eliminar contenido en un contenido tipo emisora
			\subitem Eliminar contenido en un contenido de tipo emisora
			\subitem Entrada: Petición del contenido a eliminar
			\subitem Salida: N/A
			\subitem Èxito: Emisora sin el contenido eliminado					
        \end{enumerate}			
	}			

\section{Necesidades}

\hint{Es necesario disponer de tanto de Gradle como Java >=6.}

\section{Responsabilidades}

\hint{
	\begin{enumerate}
		\item Adrián Quintás. Persona encargada de la gestión, diseño de plan de pruebas y resolución de conflitos.
		\item Yuriy Mischenko. Persona encargada de la preparación, execución y notificación del plan de pruebas.
	\end{enumerate}
}

\section{Planificación}

\hint{Una vez finalizada la primera iteraciòn del proyecto, el plan de pruebas puede ejecutarse en cualquier momento, debiendo ser satisfactorio en cualquier circunstancia.}

\section{Outros aspectos de interese}

\hint{No procede.}

\end{document}
